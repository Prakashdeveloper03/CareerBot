% Chapter 2

\chapter{\uppercase{Literature Survey/Related Work}} % Main chapter title
\label{ch:survey} % For referencing
\section{\uppercase{Overview}}

The literature survey presented in this chapter delves into various research works that are pertinent to the project's objectives, focusing on critical aspects such as recommendation system development and integration with chatbot applications. The survey covers a spectrum of topics, including item-based recommendation techniques, evaluation metrics, user engagement, and the impact of recommendation systems on career development.

\subsection{Recommender System in Academic Choices of Higher Education: A Systematic Review}

\textbf{Nabila Kamal, Farhana Sarker, Arifur Rahman, Sazzad Hossain, and Khondaker A. Mamun, 2024.} “Recommender System in Academic Choices of Higher Education: A Systematic Review". This systematic review explores the utilization of recommender systems in the context of academic choices and advising in higher education. Through the analysis of 56 primary studies published between 2011 and 2023, the study categorizes articles based on specific criteria and synthesizes their findings. It highlights the hybrid strategy as the most effective method for producing recommendations and emphasizes the importance of evaluation measures such as offline experiments and case-study validation. The review reveals that recommender system design in higher education is context-specific, with researchers considering various parameters to tailor recommendations to individual needs. However, it notes a predominance of lab-based studies over real-world applications, indicating a need for further research in practical settings. Additionally, the review identifies future research directions, including the incorporation of deep learning technologies and the analysis of personality traits. Overall, it offers valuable insights for researchers and practitioners, guiding the development of more effective and personalized recommendation systems in higher education.


\subsection{Job Recommendation System Using Hybrid Filtering}

\textbf{"Mayuresh Santosh Mhatre, Kalpesh Anil Chaudhari, Akanksha Vijay Jadhav, Vaishnavi Ashok Sathe, 2024"} “Job Recommendation System Using Hybrid Filtering", The Career Counseling web application emerges as a transformative platform tailored to assist 10th and 12th grade students in making well-informed career decisions. Addressing prevalent issues such as familial pressure and a lack of understanding leading to suboptimal career choices, this platform fills a crucial gap in current educational landscapes. By offering comprehensive information on alternate career pathways, colleges, and courses, it serves as a centralized resource for students seeking guidance. Noteworthy features include detailed college and course databases, travel directions, and interactive evaluation quizzes to gauge individual interests. Moreover, the incorporation of a dedicated forum fosters discussions on competitive exams and technical topics, enriching the learning experience. A standout feature, the College Prediction tool, aids students in selecting suitable institutions, while the Counseling Chatbot delivers real-time assistance, leveraging the power of conversational agents and Artificial Intelligence (AI). The application harnesses Machine Learning (ML) techniques, including Supervised and Unsupervised learning, to address nuanced user interactions and diverse needs effectively. Through its innovative approach, the Career Counseling web application exemplifies the potential of technology in reshaping education, empowering students, and paving the way for a brighter future.

\subsection{Edu Counselor: Career Counseling Chatbot}

\textbf{Mayuresh Santosh Mhatre, Kalpesh Anil Chaudhari, Akanksha Vijay Jadhav, Vaishnavi Ashok Sathe, 2024}
“Edu Counselor: Career Counseling Chatbot”, The integration of a unique College Prediction tool and Counseling Chatbot further enhances the user experience, providing real-time assistance and personalized recommendations. Recognizing the significance of conversational agents (CAs) in improving user experiences through Artificial Intelligence (AI), the application leverages Machine Learning (ML) methodologies, including Supervised and Unsupervised learning, to address issues in comprehending human interactions and diverse user needs. Methodologies such as User-Centric Design prioritize user research to understand specific needs, interests, and pain points, ensuring the chatbot's features are tailored accordingly. Iterative Development allows for continuous improvement based on user and expert feedback. The application utilizes modern Natural Language Processing (NLP) techniques to efficiently interpret and respond to user queries, enhancing its conversational abilities over time. Moreover, strong security measures are integrated to safeguard user data and privacy, while optimizing efficiency issues such as reaction time, scalability, data management, and user interface responsiveness.

\subsection{Personalized Career Path Recommendation Model (CPRM) for IT Job Selection}

\textbf{Puji Catur Siswipraptini, Harco Leslie Hendric Spits Warnars, Arief Ramadhan, and Widodo Budiharto, 2024} “Personalized Career Path Recommendation Model (CPRM) for IT Job Selection”. The proposed CPRM framework comprises four main layers: data acquisition, database, functional, and user application layers. The data acquisition layer collects and extracts information from various sources, while the database layer stores job, subject, and student information. The functional layer incorporates EDM-GT and recommender engine models to personalize student information and provide recommendations. The user application layer facilitates user interaction and feedback. The framework progresses through six stages, including information retrieval, construction of job and subject profiles, student profiling, dynamic CPRM mapping, and recommendation generation using the p-NB algorithm. By leveraging EDM-GT techniques, the CPRM offers dynamic and personalized recommendations, enhancing accuracy and performance in guiding students towards successful IT careers. This comprehensive framework addresses the complexities of career decision-making by integrating educational data mining and grounded theory methodologies. It represents a significant advancement in personalized career guidance within the IT domain, setting a precedent for future research and application in similar contexts.

\subsection{Applying Machine Learning Algorithm to Optimize Personalized Education Recommendation System}

\textbf{Wangmei Chen, Zepeng Shen, Yiming Pan, Kai Tan, Cankun Wang, 2024}, “Applying Machine Learning Algorithm to Optimize Personalized Education Recommendation System”. This paper delves into the evolution of education systems spurred by advancements in science and technology, particularly artificial intelligence and machine learning algorithms. Traditional education systems often lack personalization, employing a one-size-fits-all approach that overlooks individual learning needs and styles. Leveraging machine learning algorithms, personalized education systems can tailor learning materials and recommendations based on each student's history, interests, and abilities, thereby enhancing learning outcomes. Moreover, machine learning algorithms offer real-time feedback on student performance, enabling dynamic adjustments to learning plans. The applicability of such personalized systems spans various domains, including language learning, mathematics, and science. However, the efficacy of machine learning algorithms hinges on the refinement of numerical optimization algorithms. Hence, the paper provides a comprehensive overview of existing machine learning algorithms tailored for optimizing personalized education recommendation systems, elucidating the algorithm optimization process.

\subsection{Technical Job Recommendation System Using APIs and Web Crawling}

\textbf{Naresh Kumar, Manish Gupta, Deepak Sharma, and Isaac Ofori, 2024}, The research article titled “Technical Job Recommendation System Using APIs and Web Crawling” provides a comprehensive overview of the field of web crawling, which involves automatically navigating the World Wide Web to collect data from websites. The paper starts by defining web crawling and its importance in various domains such as search engines, data mining, and information retrieval. It then delves into the technical aspects of web crawling, discussing the key components involved, including URL frontier management, fetching and parsing web pages, and handling challenges such as crawler traps and politeness. The paper reviews and summarizes the existing literature on web crawling, categorizing it into different subtopics such as crawling strategies, focused crawling, distributed crawling, and crawling in specific domains like social media and deep web. It discusses various algorithms and techniques used in web crawling, such as breadth-first and depth-first crawling, link analysis, and machine learning-based approaches. The paper also addresses ethical considerations and legal issues associated with web crawling, such as respecting website owners’ terms of service and privacy concerns. Overall, this literature review serves as a valuable resource for researchers and practitioners interested in understanding the current state of web crawling and its advancements, providing insights into the challenges and potential future directions in the field.


\section{Summary of Literature Survey}

In conclusion, the literature survey encompasses a comprehensive exploration of recommendation systems for skill acquisition and career development. The reviewed journal papers provide valuable insights into item-based recommendation techniques, chatbot integration, evaluation metrics, user engagement, and the impact of recommendation systems on career development. By synthesizing these findings, the literature survey offers guidance and direction for the development of an effective recommendation system aimed at empowering individuals in their career development journey.